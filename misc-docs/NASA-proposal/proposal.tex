\documentclass[letterpaper, 10 pt, conference]{ieeeconf}  % Comment this line out if you need a4paper
%\documentclass[a4paper, 10pt, conference]{ieeeconf}      % Use this line for a4 paper

\IEEEoverridecommandlockouts                              % This command is only needed if 
% you want to use the \thanks command

\overrideIEEEmargins                                      % Needed to meet printer requirements.

\usepackage{amsmath} 
\usepackage{algorithm,algpseudocode} 
\usepackage{bm}
\usepackage{dsfont}
\usepackage{amsfonts}


\usepackage[]{hyperref} \hypersetup{ colorlinks=true, linkcolor=black, citecolor=black,
urlcolor=blue,} \usepackage{graphicx}

\def\bql{\begin{equation}} 
\def\eql{\end{equation}} 
\def\beq{\begin{equation*}}
\def\eeq{\end{equation*}} 
\def\bma{\begin{bmatrix}} 
\def\ema{\end{bmatrix}}

\DeclareMathOperator*{\argmax}{arg\,max}

%opening
    \title{Geometric Control and Estimation for Fixed-Wing UAVs}
    \author{T. Devon Morris%
    \thanks{T. Devon Morris is a PhD candidate in Electrical Engineering at Brigham Young University, devonmorris1992@byu.edu}}
\begin{document}
\maketitle 
\section{Introduction}
During the early 1960s, control theory experienced a revolution in design methodology, producing the field of modern control theory. The modern control engineer's toolbox is rooted in state space representations of systems, which rely heavily on linear algebra. Modern control has produced many incredible results in the previous half century, including the Kalman Filter, which contributed to the success of NASA's Apollo 11 mission.

The techniques from modern control are stretched and tested in systems experiencing highly nonlinear phenomena. These nonlinearities arise from two main sources:
\begin{enumerate}
  \item dynamics which depend nonlinearly on the state representation,
  \item states which are elements of a non-euclidean manifold.
\end{enumerate}
In these cases, the modern control methodology prescribes a linearization about the current state and application of well-known linear techniques: LQR, Kalman Filtering, etc. In this way, the modern control approach ignores the rich underlying structure of the manifold on which the state evolves.

Ignoring the manifold structure creates controllers and estimators that do not behave optimally. For example, on the circle a typical controller could command a rotation of $270^{\circ}$ clockwise when a rotation of $90^{\circ}$ counter-clockwise would have been more direct and efficient. These problems caused by the curvature of the manifold are exacerbated in higher dimensions. Furthermore, due to our inability to visualize manifolds higher than dimension 3, situations in which problems of this nature occur are not immediately obvious.

To solve similar nonlinear problems in physics, the mathematical field of differential geometry emerged in the early twentieth century. Differential geometry explores spaces that are globally curved but locally look like euclidean space. Using the tools of differential geometry, many local properties can be analyzed by investigating how they change along smooth curves on the manifold. The field of differential geometry has been very fruitful in explaining modern physics. 
  
The exploration of differential geometry applied to control theory, geometric control, began in the late 1980s and is still very much in its infancy. Concepts such as controllability are only narrowly defined and are still actively being explored~\cite{Bullo2004Geometric}. Geometric control has great potential to solve the shortcomings of modern control~\cite{Lewis2018Bountiful}.

NASA, as an aerospace organization, is interested in efficient and consistent control and estimation of aircraft and spacecraft. My research will focus on devloping geometric controllers and estimators that respect the manifold structure on which the pose of aircraft and spacecraft reside. Although I will focus specifically on fixed-wing estimation and control, the concepts and techniques will be generalizable to generic aircraft and spacecraft. The following sections outline the weaknesses of modern control theory and the actions I will take to incorporate geometric control.

\section{Control of Fixed-Wing UAVs}
It is well known that the state of a rigid body is an element of the non-euclidean manifold $SE(3)$, which incorporates both the position and attitude of the body. $SE(3)$ is a special type of manifold that is also a group, known as a Lie group. Lie groups exhibit symmetries and $SE(3)$ exhibits the rotational and translational symmetries present in rigid-body motion. In recent years, researchers have developed controllers for quadcopters that exploit the manifold structure of $SE(3)$~\cite{Lee2010Control, Lee2010Geometric}. The basic approach of these geometric controllers is to decouple the rotational and positional errors and represent $SE(3)$ as $SO(3) \times \mathds{R}^3$, where $SO(3)$ is the manifold representing rotations. Decoupling the errors in this fashion carries the implicit assumption that positional and rotational errors can be modeled as independent.

For fixed-wing UAVs, this approach does not work. Velocity is completely dependent of heading on fixed-wing UAVs. My research will focus on defining a controller instrinsically on $SE(3)$ that will parameterize the error natively on the manifold and its Lie algebra $\mathfrak{se}(3)$. This parameterization of error can be viewed as defining a geodesic on $SE(3)$. By following the geodesic, the controller will minimize the control effort required to command a desired pose of a fixed-wing UAV. This approach also has applications in waypoint following where a desired pose is commanded at each waypoint.

I will explore under what conditions these controllers are asymptotically stable via Lyapunov theory on manifolds. Furthermore, a true geometric controller and path planner should be coordinate independent. This is equivalent to saying that there is a canonical transformation between different sets of coordinates. I will consider these ideas of geometric invariance as I design these $SE(3)$ controllers to ensure that the controller doesn't exploit special properties of one set of coordinates.

I will also explore trajectory generation for fixed-wing UAVs on $SE(3)$ using differential geometric concepts. Typically, trajectory generation for fixed-wing UAVs is done using waypoints with the assumption that the aircraft will fly certain generic paths between these waypoints. These assumptions are insufficient when the waypoints are close together or the trajectory requires more acrobatic maneuvers. Such trajectories can be thought of embedded 1-dimensional manifolds living in $SE(3)$. Posing the problem in this manner provides the tools of differential geometry, like the covariant derivative, for motion planning~\cite{Zhang2007Robot}.


\section{Estimation of Fixed-Wing UAVs}
Estimators for fixed-wing UAVs are generally formulated in two ways: derivatives of the Extended Kalman Filter or complementary filters on the manifold.
The Multiplicative Extended Kalman Filter (MEKF) is the industry standard for attitude estimation in aerospace applications. The MEKF separates the state representation into a nominal state and an error state and gaussian statistics are defined on the error state. In this way, the MEKF respects the manifold structure of the nominal state but exploits multivariate gaussian theory for the error state. The MEKF, however, has a few shortcomings.
\begin{enumerate}
  \item The MEKF is not guaranteed to be a consistent estimator.
  \item The derivation of the dynamics of the error covariance require non-trivial assumptions on the size of the error.
  \item The Gaussian model does not represent a true probability distribution on the manifold.
\end{enumerate}
These drawbacks have strong implications that point toward the instability of the MEKF under many conditions. Other Kalman-based approaches have been derived that more fully integrate the manifold structure and can be shown to be consistent. These approaches, however, are more computationally expensive when it comes to calculating the error covariance since they require propogation of multiple points and iterative solvers~\cite{Hertzberg2013Integrating}.

Complementary filters are sometimes preferred to Kalman Filters because they exhibit nice stability properites. Most complementary filters are shown to be almost-globally exponentially stable~\cite{Lageman2010Gradient}. Furthermore, complementary filters usually have a very simple update law that makes them very computationally efficient. However, complementary filters do not have any measure of confidence or uncertainty regarding their state estimate.

Both approaches decouple rotational and positional estimation errors and, thus, lose some of the rich structure that the manifold has to offer. In exploring estimators for fixed-wing UAVs, I will first design a complementary filter natively on $SE(3)$ that exhibits almost-global stability properties. The filters created will work under several sensor modalities: GPS, motion-capture system and local maps. Once I have designed such a filter, I will investigate computationally efficient methods of parameterizing uncertainty of the estimate. Some methods to do this might be higher order complementary filters~\cite{Zlotnik2018Higher} and using probability distributions defined natively on Lie groups~\cite{Applebaum2014Probability}. As I do this, I will be careful to ensure that the estimator remains consistent and does not exhibit overconfidence. 

\section{Stability of Controller Estimator System}
As I finish the controller and estimator design, I'd like to prove that the integrated system is almost-globally asymptotically stable. In this way, the system would guarantee convergence of both the estimator and controller to the desired state given almost any initial condition. I will confirm the proof by empirically showing it converges both in simulation and hardware demonstrations.

\section{Personal Background \& Current Progress}
As a graduate researcher, I have designed and programmed multiple attitude filters for fixed-wing UAVs on $SO(3)$. I replicated the work done in~\cite{Euston2008Complementary} by designing an $SO(3)$ complementary filter and implementing it in simulation and hardware. I also designed a Multiplicative Unscented Kalman Filter on the full state of a fixed-wing UAV. Furthermore, I have modeled the command and service module from the Apollo 11 mission and have designed a geometric attitude controller based on control moment gyros.

I also recently finished a class in differential geometry, where I learned about general Riemannian manifolds and their associated affine connections (covariant derivatives). Due to my background in both mathematics and controls, I feel I am uniquely qualified to make contributions in the field of geometric control theory, especially regarding geometric control of fixed-wing UAVs.

\section{Deliverables}%
At the end of this project, I hope to deliver a few results:
\begin{enumerate}
  \item a robust almost-globally asymptotic controller for fixed-wing UAVs,
  \item a trajectory generator for acrobatic fixed-wing aircraft,
  \item a consistent, computationally efficient estimator native to $SE(3)$,
  \item empirical and mathematical proof of stability of the combined controller, estimator system.
\end{enumerate}
These results will be influential in both academia and industry for the next generation of geometric controllers and estimators.


\bibliography{MorrisBib} 
\bibliographystyle{ieeetr}

\end{document}
